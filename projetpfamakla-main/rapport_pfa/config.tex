\documentclass[12pt,a4paper]{report}

% Encodage et langue
\usepackage[utf8]{inputenc}
\usepackage[T1]{fontenc}
\usepackage[french,english]{babel} % English en dernier si on écrivait en anglais, mais ici French principal

% Mise en page
\usepackage[left=2.5cm,right=2.5cm,top=2.5cm,bottom=2.5cm]{geometry}
\usepackage{setspace}
\onehalfspacing % Interligne 1.5

% Graphiques et Figures
\usepackage[demo]{graphicx} % [demo] pour compiler sans les images (remplace par des carrés noirs)
\usepackage{float}
\usepackage{caption}
\usepackage{subcaption}
\graphicspath{{images/}}

% Liens et références
\usepackage[colorlinks=true, linkcolor=blue, citecolor=green, urlcolor=cyan]{hyperref}

% Mathématiques
\usepackage{amsmath, amssymb, amsfonts}

% Code source
\usepackage{listings}
\usepackage{xcolor}
\definecolor{codegreen}{rgb}{0,0.6,0}
\definecolor{codegray}{rgb}{0.5,0.5,0.5}
\definecolor{codepurple}{rgb}{0.58,0,0.82}
\definecolor{backcolour}{rgb}{0.95,0.95,0.92}

\lstdefinestyle{mystyle}{
    backgroundcolor=\color{backcolour},   
    commentstyle=\color{codegreen},
    keywordstyle=\color{magenta},
    numberstyle=\tiny\color{codegray},
    stringstyle=\color{codepurple},
    basicstyle=\ttfamily\footnotesize,
    breakatwhitespace=false,         
    breaklines=true,                 
    captionpos=b,                    
    keepspaces=true,                 
    numbers=left,                    
    numbersep=5pt,                  
    showspaces=false,                
    showstringspaces=false,
    showtabs=false,                  
    tabsize=2
}
\lstset{style=mystyle}

% En-têtes et pieds de page
\usepackage{fancyhdr}
\pagestyle{fancy}
\fancyhf{}
\fancyhead[L]{\leftmark}
\fancyhead[R]{\thepage}
\renewcommand{\headrulewidth}{0.4pt}
%\renewcommand{\footrulewidth}{0.4pt}

% Bibliographie
\usepackage[backend=biber,style=numeric,sorting=none]{biblatex}
\addbibresource{bibliographie.bib}

% Titres des chapitres
\usepackage{titlesec}
\titleformat{\chapter}[display]
  {\normalfont\huge\bfseries}{\chaptertitlename\ \thechapter}{20pt}{\Huge}

% Lorem ipsum pour le remplissage
\usepackage{lipsum}

% Commandes personnalisées
\newcommand{\univName}{NOM DE L'UNIVERSITÉ}
\newcommand{\deptName}{DÉPARTEMENT D'INFORMATIQUE}
\newcommand{\reportTitle}{TITRE DU PROJET PFA}
\newcommand{\studentName}{Nom Prénom de l'Étudiant}
\newcommand{\supervisorName}{Nom Prénom de l'Encadrant}
